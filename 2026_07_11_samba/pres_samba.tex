\documentclass[10pt]{beamer}
\usepackage{beamerbasenotes}
\usepackage{appendixnumberbeamer}
\usepackage{hyperref}
\usepackage[utf8x]{inputenc}
\usepackage{graphicx}
\usepackage[nodayofweek]{datetime}
\usepackage[colorinlistoftodos]{todonotes}
\usepackage{graphicx}
\usepackage{mathtools}
\usepackage{amssymb}
\usepackage{amsmath}
\usepackage{bbm}
\usepackage{pdflscape}
\usepackage{caption}
\usepackage{subcaption}
\usepackage[T1]{fontenc}
\usepackage{authblk}
\usepackage{pdfpages}
\usepackage{setspace} 
\usepackage{booktabs}
\usepackage{longtable}
\usepackage{float}
\usepackage{tikz}
\usepackage{bm}
\usepackage{appendixnumberbeamer}
\usepackage{siunitx}
% \usepackage{natbib}
\usepackage[style=authoryear,backend=biber]{biblatex}
\usepackage{bibentry}

% \nobibliography*
\addbibresource{library.bib}

% ------------------------------------------------------------------------------
% Use the beautiful metropolis beamer template
% ------------------------------------------------------------------------------
\usepackage[T1]{fontenc}
\usepackage{fontawesome}
\usepackage{FiraSans} 
\usepackage{fontspec}
\setmonofont{Fira Mono}
\mode<presentation>
{
  \usetheme[progressbar=foot,numbering=fraction,background=light]{metropolis} 
  \usecolortheme{default} % or try albatross, beaver, crane, ...
  \usefonttheme{default}  % or try serif, structurebold, ...
  \setbeamertemplate{navigation symbols}{}%[appendixframenumber]
  % \setbeamertemplate{caption}[numbered]
  %\setbeamertemplate{frame footer}{My custom footer}
} 

\setbeamertemplate{section in toc}{\hspace*{1em}\inserttocsectionnumber.~\inserttocsection\par}
\setbeamertemplate{subsection in toc}{\hspace*{2em}\inserttocsectionnumber.\inserttocsubsectionnumber.~\inserttocsubsection\par}

\setbeamertemplate{footnote}{%
  \parindent 1em\noindent%
  \raggedright
  \tiny\insertfootnotetext\par%
}

\makeatletter
\setbeamertemplate{title page}{
  \vbox{}
  \vfill
  \begin{centering}
    {\usebeamerfont{title}\inserttitle\par}
    \vskip0.5em
    {\usebeamerfont{author}\insertauthor\par}
    \vskip0.5em
    {\usebeamerfont{date}\insertdate\par}
    \vskip1em
    \inserttitlegraphic  % << this places it BELOW the title
  \end{centering}
  \vfill
}
\makeatother

\makeatletter
\addtobeamertemplate{footline}{%
  \vskip1ex%
  {\usebeamerfont{footnote}%
   \usebeamercolor[fg]{footnote}%
   \insertfootnotetext\par}%
}{}
\makeatother

% ------------------------------------------------------------------------------
% beamer doesn't have texttt defined, but I usually want it anyway
% ------------------------------------------------------------------------------
\let\textttorig\texttt
\renewcommand<>{\texttt}[1]{%
  \only#2{\textttorig{#1}}%
}

% ------------------------------------------------------------------------------
% minted
% ------------------------------------------------------------------------------
\usepackage{minted}


% ------------------------------------------------------------------------------
% tcolorbox / tcblisting
% ------------------------------------------------------------------------------
\usepackage{xcolor}
\definecolor{codecolor}{HTML}{FFC300}

\usepackage{tcolorbox}
\tcbuselibrary{most,listingsutf8,minted}

\tcbset{tcbox width=auto,left=1mm,top=1mm,bottom=1mm,
right=1mm,boxsep=1mm,middle=1pt}

\newtcblisting{myr}[1]{colback=codecolor!5,colframe=codecolor!80!black,listing only, 
minted options={numbers=left, style=tcblatex,fontsize=\tiny,breaklines,autogobble,linenos,numbersep=3mm},
left=5mm,enhanced,
title=#1, fonttitle=\bfseries,
listing engine=minted,minted language=r}


% ------------------------------------------------------------------------------
% Listings
% ------------------------------------------------------------------------------
\definecolor{mygreen}{HTML}{37980D}
\definecolor{myblue}{HTML}{0D089F}
\definecolor{myred}{HTML}{98290D}

\usepackage{listings}

% the following is optional to configure custom highlighting
\lstdefinelanguage{XML}
{
  morestring=[b]",
  morecomment=[s]{<!--}{-->},
  morestring=[s]{>}{<},
  morekeywords={ref,xmlns,version,type,canonicalRef,metr,real,target}% list your attributes here
}

\lstdefinestyle{myxml}{
language=XML,
showspaces=false,
showtabs=false,
basicstyle=\ttfamily,
columns=fullflexible,
breaklines=true,
showstringspaces=false,
breakatwhitespace=true,
escapeinside={(*@}{@*)},
basicstyle=\color{mygreen}\ttfamily,%\footnotesize,
stringstyle=\color{myred},
commentstyle=\color{myblue}\upshape,
keywordstyle=\color{myblue}\bfseries,
}

% Need valign for slide with equation and image
\usepackage{adjustbox}

% ------------------------------------------------------------------------------
% The Document
% ------------------------------------------------------------------------------

\titlegraphic{%
  \centering
  \includegraphics[height=1.5cm]{images/university-of-bath-logo.png}\hspace{1cm}%
  \includegraphics[height=1.5cm]{images/UKRIlogo.png}%
}

% do an un-numbered \againframe by bumping the counter back
\newcommand{\unnumberedagain}[2]{%
  \addtocounter{framenumber}{-1}%  remove the extra count
  \againframe<#1>{#2}%
}

\begin{document}

\title{A clustering framework for conditional extremes models}

% \newdate{date}{14}{05}{2025}
\newdate{date}{11}{07}{2025}
\date{
  % \footnotesize Conference on Applied Statistics in Ireland (CASI) 2025 Talk, \\
  % \footnotesize Environmental and Ecological Statistics Conference 2025 Talk, \\
  \footnotesize SAMBa Conference 2025 Talk, \\
  \displaydate{date}, \\
  \emph{Paddy O'Toole, University of Bath} \\
  \emph{Supervised by Christian Rohrbeck and Jordan Richards (University of Edinburgh)}
}

\begin{frame}[noframenumbering, plain]
  \titlepage
\end{frame}

% \begin{frame}
%   \addtocounter{framenumber}{-1}
%   % \item Get down to 20ish slides before Christian meeting
%   % \item Cut down on words wherever possible
%   % \item Do TODOs
%   \item Add QR code slide
%   % \item Fix clipping plots (requires saving manually!) (including appendix!)
%   % \item Check slide titles, don't capitalise
%   % \item Swap pdfs for pngs
%   % \item Finalise plots
%   % \item Add appendix slides + plots
%   % \item Remove commented slides when finished
%   \item Write script
%   % Full presentation notes
%   % \item Remove references to GPD
%   % \item Update simulation slides/plots (or don't)
%   % \item Go through Christians notes on slides
%   % \item Cut down on words
%   % \item Update chi plot
%   % \item Remove medians from clustering map
%   % \item Add diagnostic plots
%   % \item Comment (and do) todos (temporarily)
% \end{frame}

%=================================================

% \begin{frame}
% \frametitle{Table of contents}
% \tableofcontents
% \end{frame}

\section{Introduction}


% Images of Pulteney Bridge Weir and plaque
% Frank Greenhalgh, Engineer who designed Pulteney Weir and Sluce for Bristol Avon River Authority
\begin{frame}
   \begin{minipage}{0.5\textwidth}
     \begin{center}
       \hspace*{-0.7cm} \includegraphics[width=1.4\textwidth]{images/pultney-bridge.jpg}
     \end{center}
   \end{minipage}
   \hfill
   \begin{minipage}{0.4\textwidth}
     \hspace*{0.8cm} \includegraphics[width=1\textwidth]{images/Hapennybr_lower.jpg} 
   \end{minipage}
\end{frame}

\begin{frame}{Problem} 
  \begin{itemize}
    \item Often want to estimate
      \[
        \mathbb{P}(X > x, Y > y) = \textcolor{red}{\mathbb{P}(Y > y \mid X > x)}\textcolor{blue}{\mathbb{P}(X > x)}
      \]
      for large $x, y$
    \item ``concomitant''/concurrent extreme events for random vector $\bf{X}$ often particularly devastating
    \item Goal: \textbf{identify trends by clustering sites with similar \textcolor{red}{tail dependence}}
  \end{itemize}
\end{frame}

\section{Dependence Modelling}

\begin{frame}[fragile]{Asymptotic dependence}
  % reserve space for your citation block
  \newlength{\TopBoxHeight}
  \setlength{\TopBoxHeight}{\dimexpr\textheight-3cm\relax}

  \vbox to \TopBoxHeight{
    \vfill
    \begin{center}
      \begin{itemize}
        % \item Asymptotic dependence is a measure of the strength of dependence between two random variables in the tail of their distribution.
        \item Coefficient of extremal dependence $\chi\in[0,1]$, 
              \[
                \chi = \lim_{u\to1}\mathbb{P}\bigl[F_1(X_1)>u\mid F_2(X_2)>u\bigr]
              \]
        \item (Increasingly strong) asymptotic dependence for $\chi > 0$.
        \item However, $\chi$ only gives summary; inference requires \textbf{dependence model}.
      \end{itemize}
    \end{center}
    \vfill
  }

  \vspace{1.5cm}
  {\scriptsize
    \rule{\linewidth}{0.4pt}\\
    \makebox[\linewidth]{
      \citeauthor{Coles1999}, \emph{\citetitle{Coles1999}} (1999)
    }
  }
\end{frame}

\begin{frame}{Ireland}
  \begin{minipage}{0.4\textwidth}
      \begin{itemize}
        \item Precipitation\textsuperscript{1} \& wind speed\textsuperscript{2} data for 59 sites across Ireland, Winter months (Oct-Mar) 1990-2020
      \end{itemize}
  \end{minipage}
  \begin{minipage}{0.59\textwidth}
    \includegraphics[width = \linewidth]{plots/041_chi_plots_man_crop.png}
  \end{minipage}
  {\scriptsize
    \rule{\linewidth}{0.4pt}\\
    \makebox[\linewidth]{
      \textsuperscript{1} Met Éireann weekly aggregate
      \hfill{}
      \textsuperscript{2} ERA5 reanalysis weekly mean of daily maxima
      \hspace{0.5cm}
    }
  }
\end{frame}

\section{Conditional extremes}

\begin{frame}{Marginal transformation}
  % allows for heteroskedastic regression model, symmetry means model is the same
  % for both positive and negative dependence, and also easy to model any
  % dependence strength
  \center
  \includegraphics[width=0.9\textwidth]{plots/transform_costelloe.png} 
  % {\scriptsize
  %   \rule{\linewidth}{0.4pt}\\
  %   \makebox[\linewidth]{
  %     \citeauthor{Keef2013}, \emph{\citetitle{Keef2013}} (2013)
  %   }
  % }
\end{frame}

% Multivariate piecewise function, Z, Y independent
\begin{frame}[t]{Conditional extremes}
% Reduces multivariate model to simple univariate models!
\begin{itemize}
  \item \textbf{Heteroskedastic regression dependence model}: 
  \begin{equation*}
    \bigl(Y \mid X = x\bigr) = \alpha x + x^{\beta} Z, \text{ for } x > u
  \end{equation*}
  \vspace{-5mm}
  % with
  \begin{itemize}
    \item slope parameter $\alpha$ for $Y$ given large $X$,
    \item ``spread'' parameter $\beta \in (-\infty, 1]$ controls stochasticity of relationship between $Y$ and large $X$. 
  \end{itemize}
\end{itemize}
\end{frame}

\begin{frame}[t, noframenumbering]{Conditional extremes}
% Reduces multivariate model to simple univariate models!
\begin{itemize}
  \item <1-> \textbf{Heteroskedastic regression dependence model}: 
  \begin{equation*}
      \bigl(\bm{Y}_{-i} \mid Y_i = y_i \bigr) = \bm{\alpha}_{\mid i}\bm{y}_i + \bm{y}_i^{\bm{\beta}_{\mid i}}\bm{Z}_{\mid i}, \text{ for } y_i > u_{Y_i}
  \end{equation*}
  \vspace{-5mm}
  % with
  \begin{itemize}
    \item <1-> slope parameter $\alpha_{j \mid i} \in [-1, 1]$  for $Y_j$ given large $Y_i$,
    \item <1-> ``spread'' parameter $\beta_{j \mid i} \in (-\infty, 1]$ controls stochasticity of relationship between $Y_j$ and large $Y_i$. 
  \end{itemize}
  \vspace{2mm}
  \item <2-> Key assumptions:
  \begin{itemize}
    \item <2-> Residuals $\bm{Z}_{\mid i} \sim N(\bm{\mu}_{\mid i}, \bm{\Sigma}_{\mid i})$ 
    \item <2-> $\bm{Z}_{\mid i}, Y_i$ conditionally independent for large $Y_i$
  \end{itemize}
  \vspace{2mm}
  % \item <2-> Special cases:
  % \vspace{-4mm}
  % \begin{alignat*}{2}
  % &\bm{\alpha}_{\mid i} = 0, \bm{\beta}_{\mid i} = 0    &&\implies \bm{Y}_{-i}, \bm{Y}_{i} \text{ independent,} \\
  % &\bm{\alpha}_{\mid i} = -1/1, \bm{\beta}_{\mid i} = 0 &&\implies \text{ perfect positive/negative dependence,} \\
  % &-1 < \bm{\alpha}_{\mid i} < 1 && \implies \text{ asymptotic independence}
  % \end{alignat*}
\end{itemize}
\vspace{1.3cm}
{\scriptsize
  \rule{\linewidth}{0.4pt}\\
  \makebox[\linewidth]{
    \citeauthor{Heffernan2004}, \emph{\citetitle{Heffernan2004}} (2004)
  }
}
\end{frame}

\begin{frame}{Conditional extremes}
  \center{}
  \includegraphics[width=\textwidth]{plots/041_costelloe_fishery.png} 
\end{frame}

\begin{frame}{Inference}
  \begin{itemize}
    Inference assumes conditional distribution follows a multivariate Normal (MVN) distribution: \\
      \begin{equation*} \label{eq:ce_norm}
        \bigl( \bm{Y}_{-i} \mid Y_i = y_i \bigr) \sim N\left(\bm{\alpha}_{\mid i} y_i + y_i^{\bm{\beta}_i} \bm{\mu}_{\mid i}, y_i^{\bm{\beta}_i} \bm{\Sigma_{\mid i}}\right), \text{ for } Y_i > u_{Y_i}
      \end{equation*} \\
    $\implies$ \textbf{dependence structures at different sites can be compared using their MVN distributions}
\end{itemize}
\end{frame}

\section{Clustering}

% \begin{frame}{skew-geometric Jensen-Shannon divergence}
\begin{frame}{KL Divergence}
  \begin{itemize}
    \item <1-> A popular form of statistical distance is the Kullback-Leibler divergence % measures (\textbf{asymmetric}) distance from models $P$ to $Q$
    \[
    KL(P \mid\mid Q) = \int_{-\infty}^{\infty} p(y) \log\left(\frac{p(y)}{q(y)}\right) \mathrm{d}y.
    \]
    % triangle inequality: “direct” distance between two distributions is never larger than going through a third.
    \item <2-> $KL(P \mid \mid Q)$ has closed form for two MVNs
    \item <3-> Why not $KL(P \mid\mid Q) + KL(Q \mid\mid P)$? % symmetric but does not satisfy triangle inequality $\implies$ \textbf{not a true metric}.
    % \item <2-> Weighted geometric mean $G_{\alpha}(P, Q) = P^{\alpha} Q^{1 - \alpha}, \alpha \in [0, 1]$
    % \item <3-> weighted product $G_{\alpha}$ of two exponential family members is in \textbf{same family}.
    % \item <4-> \textbf{Skew-geometric Jensen-Shannon divergence}:
    %   \[
    %     JS^{G_{\alpha}}(P \mid\mid Q) = \frac{1}{2} \left\{KL(P \mid\mid G_{\alpha}(P, Q)) + KL(Q \mid\mid G_{\alpha}(P, Q))\right\}
    %   \]
    % \item <5-> P, Q depend on X $\implies$ we define the expected value
    %   \[
    %     \int_{u_X}^\infty JS^{G_{\alpha}}(P(x) \mid\mid Q(x)) f(x) \mathrm{d}x,
    %   \]
    %   where $f(x)$ is the marginal density of $X \mid X > u_X$.
  \end{itemize}
  % \vspace{0.5cm}
  % {\only<4>{\scriptsize
  %   \rule{\linewidth}{0.4pt}\\
  %   \makebox[\linewidth]{
  %     \citeauthor{Nielsen2019}, \emph{\citetitle{Nielsen2019}} (2019)
  %   }
  % }}
\end{frame}

\begin{frame}{Skew-geometric Jensen-Shannon divergence}
  \begin{itemize}
    \item <1-> Weighted geometric mean $G_{\alpha}(P, Q) = P^{\alpha} Q^{1 - \alpha}, \alpha \in [0, 1]$
    % \item <2-> weighted product $G_{\alpha}$ of two exponential family members is in \textbf{same family}.
    \item <2-> weighted product of two exponential family members is in \textbf{same family}.
    \item <3-> \textbf{Skew-geometric Jensen-Shannon divergence}:
      \[
        JS^{G_{\alpha}}(P \mid\mid Q) = \frac{1}{2} \left\{KL(P \mid\mid G_{\alpha}(P, Q)) + KL(Q \mid\mid G_{\alpha}(P, Q))\right\}
      \]
    \item <4-> P, Q depend on X $\implies$ we define the expected value
      \[
        \int_{u_X}^\infty JS^{G_{\alpha}}(P(x) \mid\mid Q(x)) f(x) \mathrm{d}x,
      \]
      where $f(x)$ is marginal density of $X \mid X > u_X \sim Exp(1) + u_X$ 
  \end{itemize}
  \vspace{0.5cm}
  {\only<4>{\scriptsize
    \rule{\linewidth}{0.4pt}\\
    \makebox[\linewidth]{
      \citeauthor{Nielsen2019}, \emph{\citetitle{Nielsen2019}} (2019)
    }
  }}
\end{frame}

\begin{frame}{Clustering}
  \begin{itemize}
    \item Use \textbf{k-medoids} to cluster over pairwise $JS_{G_{\alpha}}$ dissimilarity matrix between sites
    \item \textbf{Elbow plot}: $k$ chosen using \textbf{Total Within Group Sum of Squares} between sites $x$ within clusters $C_i$ with medoids $m_i$:
     \[
       \mathrm{TWGSS}(k) = \sum_{i=1}^{k}\sum_{x \in C_i} JS^{G_{\alpha}}\bigl(x \mid \mid m_i\bigr)
     \]
  \end{itemize}
  % \vspace{0.1em}
  % \vspace{-2em}
  \begin{center}
    \vspace{-0.3cm}
    \includegraphics[width=0.6\textwidth]{plots/elbow_plot.png}
  \end{center}
\end{frame}

% \begin{frame}{Example}
% \todo{Add plot of Gaussian data}
% Plot of Gaussian data
% \end{frame}

\begin{frame}{Steps}
  \begin{enumerate}
    \item <1-> Transform to Laplace margins
    \item <2-> Fit CE model to estimate $\bm{\alpha}, \bm{\beta}, \bm{\mu}, \Sigma$ at every site. % for $Y_1 \mid Y_2, Y_2 > u_{Y_2}$ and $Y2 \mid Y1, Y1 > u_{Y_1}$.
    \item <3-> Estimate $N\left(\bm{\alpha} y_i + y_i^{\bm{\beta}} \bm{\mu}, y_i^{\bm{\beta}}\Sigma \right)$
    \begin{itemize}
      \item <3-> $y_i \sim \text{Exponential}(1) + u_{Y}$
      % \item <4-> As an example, take sites 1 \& 3 and condition on $Y2$:
      % \begin{align*}
      %      (Y_1 \mid Y_2)_1 &\sim N((0.31, 0.47, 0.33, \ldots), \diag(1.55, 1.45, 1.53, \ldots)), \\
      %      (Y_1 \mid Y_2)_3 &\sim N((2.84, 2.56, 3.35, \ldots), \diag(0.94, 0.89, 1.04, \ldots))
      % \end{align*}
    \end{itemize}
    \item <4-> Calculate pairwise $JS^{G_{\alpha}}$ dissimilarity between sites for each variable
    \item <4-> Cluster!
\end{enumerate}
\end{frame}

% \begin{frame}{Laplace Data}
% Plot of tranformed data
% \end{frame}
% 
% \againframe<3>{stepslide}
% 
% \begin{frame}{CE parameters}
% Table of fitted CE parameter values
% \end{frame}
% 
% \againframe<5>{stepslide}
% 
% % \begin{frame}{$JS^{G_{\alpha}$}
% \begin{frame}{Dissimilarity matrices}
% Dissimilarity matrices
% \end{frame}

% 1. First show only step 1 of the steps slide:
% \begin{frame}<1>[label=stepslide]{Analysis Steps}
%   \begin{enumerate}
%     \item<1-> Transform to Laplace margins
%     \item<2-> Fit CE model to estimate parameters
%     \item<3-> Calculate site‐specific distributions
%     \item<4-> Compute JS dissimilarity matrices
%   \end{enumerate}
% \end{frame}

% Second try
% \begin{frame}<1->[label=stepslide]{Steps}
%   \begin{enumerate}
%     \item <1--> Transform to Laplace margins
%     \item <3--> Fit CE model to estimate $\alpha, \beta, \mu, \sigma$ values for $Y_1 \mid Y_2, Y_2 > u_{Y_2}$ and $Y2 \mid Y1, Y1 > u_{Y_1}$.
%     \item <5--> Calculate $N\left(\alpha y_i + y_i^{\beta} \mu, y_i^{\beta}\sigma \right)$ at each individual site for both models $i \in [1, 2]$.
%     \begin{itemize}
%       \item <6--> Since Laplace has exponential tails, $y_i \sim \text{Exponential}(1) + \max{(u_{Y_i})}$
%       \item <6--> As an example, take sites 1 \& 3 and condition on $Y2$:
%       \begin{align*}
%            (Y_1 \mid Y_2)_1 &\sim N((0.31, 0.47, 0.33, \ldots), \diag(1.55, 1.45, 1.53, \ldots)), \\
%            (Y_1 \mid Y_2)_3 &\sim N((2.84, 2.56, 3.35, \ldots), \diag(0.94, 0.89, 1.04, \ldots))
%       \end{align*}
%     \end{itemize}
%     \item <7--> Calculate pairwise $JS^{G_{\alpha}}$ between MVNs across all sites for $Y1 \mid Y2$ and $Y2 \mid Y3$ to get dissimilarity matrices
% \end{enumerate}
% \end{frame}
% 
% % 2. Show the Laplace Data plot right after step 1
% \begin{frame}[noframenumbering]{Laplace Data}
%   your plot here, e.g.
%   % \includegraphics[width=\textwidth]{laplace_data_plot.pdf}
% \end{frame}
% 
% % 3. Go back to the “Steps” slide, now revealing step 2
% % \againframe[noframenumbering]<2>{stepslide}
% \unnumberedagain{2}{stepslide}
% 
% % 4. Show the CE parameters slide after step 2
% \begin{frame}[noframenumbering]{CE Parameters}
%   % your table or content here
% \end{frame}
% 
% % 5. Back to “Steps” revealing step 3
% % \againframe[noframenumbering]<3>{stepslide}
% \unnumberedagain{3}{stepslide}
% 
% % 6. Show the Dissimilarity Matrices slide after step 3
% \begin{frame}{Dissimilarity Matrices}
%   % your matrices here
% \end{frame}
% 
% % 7. Final return to “Steps” revealing step 4
% % \againframe[noframenumbering]<4>{stepslide}
% \unnumberedagain{4}{stepslide}

\section{Simulations}

\begin{frame}{Gaussian copula}
  \todo{Stop plot clipping}
  \begin{itemize}
    \item For single ``site'', generate data from bivariate Gaussian copula with correlation $\rho_{\text{Gauss}}$.
    \item (asymptotic) CE parameters are $\alpha = \rho_{\text{Gauss}}^2, \beta = 1/2$
    \item Example: 12 ``sites'' with \num[group-separator={,}]{10000} observations of 2 variables
    \item 3 clusters of 4 locations defined by respective $\rho_{\text{Gauss}}$ values of $0.1, 0.5, 0.9$.
  \end{itemize}
\end{frame}

\begin{frame}{Gaussian copula}
  \center
  \includegraphics[width=\textwidth]{plots/gauss_sim_ex_man.png}
\end{frame}

\begin{frame}{Gaussian copula - results}
  \center
  \includegraphics[width=\textwidth]{plots/sim_01_gauss_cop_man.png}
\end{frame}

\begin{frame}{Mixture simulation}
  \begin{itemize}
    \item <1-> Extend design to mixture of Gaussian and t-copulas
    \item <2-> Idea: 
    \begin{itemize} 
      \item Gaussian copula generates observations exhibiting extremal independence, 
      \item t-copula induces extremal dependence.
    \end{itemize}
    \item <3-> Design: 12 sites with \num[group-separator={,}]{10000} bivariate observations
    \item <3-> 2 clusters each of 6 sites
    \item <4-> Grid search performed over $\rho_{\text{Gauss}}, \rho_{t_1}, \rho_{t_2} \in [0, 1]$.
    \item <5-> Clustering compared to competing Vignotto algorithm using Adjusted Rand Index $\text{ARI} \in [-0.5, 1]$ 
  \end{itemize}

  {\only<5>{\scriptsize
    \rule{\linewidth}{0.4pt}\\
    \begin{minipage}{\linewidth}
      \vspace{2mm}
      \citeauthor{Vignotto2021},\\
      \emph{\citetitle{Vignotto2021}} (2021)
    \end{minipage}
  }}
\end{frame}

% \begin{frame}{Mixture simulation}
%   \center
%   \includegraphics[width=1\textwidth]{plots/t_sim_ex.png}
% \end{frame}

\begin{frame}{Mixture simulation}
  \center
  \includegraphics[width=1\textwidth]{plots/sim_01b_ce_vs_vi_dqu_0.9_man.png}
\end{frame}

\begin{frame}{Mixture simulation}
  \center
  \includegraphics[width=0.95\textwidth]{plots/sim_01d_js_sens_3_var_dqu_0.9mult_var_man.png}
\end{frame}

\section{Ireland meteorological data}

% Maps of clustering
\begin{frame}{Ireland clustering}
  \center
  \includegraphics[width=0.9\textwidth]{plots/cluster_dqu_sep.png} 
\end{frame}

\section{Discussion}

\begin{frame}{Discussion}
  \begin{itemize}
    \item <1-> \textbf{Conclusion}:
      \begin{itemize}
        \item Principled \& effective clustering framework for CE models
        \item Simulations \& application show clustering can group sites with similar tail dependence, which may aid interpretation
      \end{itemize} 
    \vspace{2mm}
    \item <2-> \textbf{Limitations}:
      \begin{itemize}
        \item \textbf{CE}: Gaussian assumption for $\bm{Z}_{\mid i}$
        \item \textbf{Clustering}: Uncertainty in CE fits is ignored in clustering
      \end{itemize} 
  \end{itemize} 
\end{frame}

% QR Code slide
\begin{frame} 
  \addtocounter{framenumber}{-1}
  % Thank you! For slides and supplementary materials, please scan the QR code below:
  Thank you! For slides and supplementary materials, please scan the QR code below:
  \begin{center}
    \includegraphics[width=0.5\textwidth]{images/9pW0kU.jpg}
  \end{center}
  Email: \url{pot23@bath.ac.uk} 
\end{frame}

\appendix

\begin{frame}\frametitle{References}
\tiny
\printbibliography[heading=none]
\end{frame}

\section{Appendix}

\begin{frame}{Mixture simulation}
  \center
  \includegraphics[width=1\textwidth]{plots/t_sim_ex.png}
\end{frame}

\begin{frame}{Mixture simulation (three variables)}
  \center
  \includegraphics[width=1\textwidth]{plots/sim_01d_js_sens_3_var_dqu_0.9_man.png}
\end{frame}

\begin{frame}{More realistic example}
  \center
  \includegraphics[width=1\textwidth]{plots/01c2_js_sens_sim_60_locs_dqu_0.9_smooth_man.png}
\end{frame}

\begin{frame}{Uncertainty Reduction}
  \center
  \includegraphics[width=1\textwidth]{plots/sim_01e_bootstrap_man.png}
\end{frame}

\begin{frame}{$\alpha, \beta$ pre-clustering}
  \center
  \includegraphics[width=0.75\textwidth]{plots/ire_ce.png} 
\end{frame}

\begin{frame}{$\alpha, \beta$ post-clustering}
  \center
  \includegraphics[width=0.72\textwidth]{plots/ab_map_post_clust.png}
\end{frame}

\end{document}

